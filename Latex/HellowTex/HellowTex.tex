\documentclass[UTF8]{article}  
\title{Data mining}
\author{Prince}
\date{\today}




\begin{document}        


\maketitle


\begin{abstract}
\emph{Data mining} --  an interdisciplinary subfield of computer science. It is the computational process of discovering patterns in large data sets involving methods at the intersection of artificial intelligence, machine learning, statistics, and database systems. The overall goal of the data mining process is to extract information from a data set and transform it into an understandable structure for further use. Aside from the raw analysis step, it involves database and data management aspects, data pre-processing, model and inference considerations, interestingness metrics, complexity considerations, post-processing of discovered structures, visualization, and online updating. Data mining is the analysis step of the "knowledge discovery in databases" process, or KDD\footnote{What is KDD is to be explained here}.
\end{abstract}

\begin{quote}
this is quotations
\end{quote}
\section{Introduction} 

The actual data mining task is the automatic or semi-automatic analysis of large quantities of data to extract previously unknown, interesting patterns such as groups of data records (cluster analysis), unusual records (anomaly detection), and dependencies (association rule mining). This usually involves using database techniques such as spatial indices. These patterns can then be seen as a kind of summary of the input data, and may be used in further analysis or, for example, in machine learning and predictive analytics. For example, the data mining step might identify multiple groups in the data, which can then be used to obtain more accurate prediction results by a decision support system. Neither the data collection, data preparation, \emph{this is a formula in lines} $a^2+b^2=c^2$ nor result interpretation and reporting is part of the data mining step, but do belong to the overall KDD process as additional steps .
\[
    a(a+3)+55^a+(a+b)=b
\]

\[
    a^2 + b^2 = c^2         
\]
\bibliographystyle{math}
\end{document}
